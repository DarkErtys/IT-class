\documentclass[a4paper,12pt]{article}

\usepackage{latexsym}
\usepackage[utf8]{inputenc}
\usepackage[T1]{fontenc}
\usepackage{graphicx}
\usepackage{float}
\usepackage{listings}
\usepackage{textcomp}
\usepackage{hyperref}

\author{Krzysztof~Palka}

\title{\textsc{Exercise} 3 \\ Linux installation} 

\addtolength{\textwidth}{2.5cm}
\addtolength{\hoffset}{-1.25cm}


\begin{document}

    \maketitle

    \lstset{showspaces=false,
            showlines=false, 
            language=sh,
            numbers=left,
            firstnumber=last,
            breaklines=true,
            upquote=true}

    \begin{abstract}
        This report presents process of installation and configuration of Linux system on virtual machine. 
    \end{abstract}

	\section{Introduction}
    I will use VirtualBox program as virtual machine and Fedora 32-bit Linux distribution. VirtualBox packages are are available at \url{https://www.virtualbox.org/wiki/Downloads} and List of Fedora packages here: \url{http://mirrors.fedoraproject.org/publiclist/Fedora/18/}. I'm downloading i386 distribution which stands for 32 bit version.  

    \section{Preparation of virtual machine}
    I'm installing VirtualBox and creating new machine with following parameters
    \begin{description}
        \item[Name] Fedora
        \item[Memory] 768 MB
        \item[Disk file type] VDI
        \item[Disk storage details] Dynamically allocated
        \item[Disk size] 10 GB
    \end{description}
    
    I'm setting downloaded iso file as start disk and starting machine.
    \section{Installation}
    I'm choosing \emph{Install Fedora}. During verification of installation media I'm pressing \emph{Esc}, to skip this proces.
    I'm choosing default language and going to partitioning section by clicking on disk icon. As \emph{Partition type} I'm choosing \emph{Standard Partition} and selecting \emph{Let me customize the partitioning of the disks instead.}    
    I'm adding new partition---root---by typing \emph{/} as \emph{Mount Point} and \emph{2GB} as \emph{Desired capacity}. 
    In \emph{customize} section I'm choosing: 
    \begin{description}
        \item[Device Type] Standard Partition
        \item[File System] ext4
    \end{description}
    
    And applying changes. In analogous way I'm adding following partitions:
    \begin{description}
        \item [/boot] 500 MB
        \item [/home] 500 MB
        \item [/tmp] 500 MB
        \item [/usr] 4 GB
        \item [/var] 2 GB
    \end{description}
    And finally I'm adding swap partition for remaining space by typing \emph{swap} in \emph{Mount Point} and leaving \emph{Desired capacity} field empty.
    I'm clicking \emph{Finish Partitioning} and then \emph{Software selection}. As environment I'm choosing \emph{Minimal Install}, clicking \emph{Done} and \emph{Begin Installation}.
    I'm setting root password and finishing configuration and clicking \emph{reboot}. After reboot of VM I'm unchecking Fedora installation iso file (Devices $\rightarrow$ CD/DVD) and rebooting it again. 

    When system boot I'm logging in as \emph{root} and typing following commands: 
    \begin{lstlisting}[frame=single]
cat /proc/mounts
tune2fs -c 10 -i 2m -e remount-ro /dev/sda1
tune2fs -c 10 -i 2m -e remount-ro /dev/sda2
tune2fs -c 10 -i 2m -e remount-ro /dev/sda3
tune2fs -c 10 -i 2m -e remount-ro /dev/sda4
tune2fs -c 10 -i 2m -e remount-ro /dev/sda5
tune2fs -c 10 -i 2m -e remount-ro /dev/sda6
tune2fs -c 10 -i 2m -e remount-ro /dev/sda7
tune2fs -l /dev/sda1
    \end{lstlisting}
    cat /proc/mounts shows current mount options. tune2fs is used to adjust tunable filesystem parameters on filesystems. I used following arguments: 

    \begin{description}
        \item [-c 10] Adjust the maximal mounts count between two filesystem checks to 10.
        \item [-i 2m] Adjust the maximal time between two filesystem checks to 2 months.
        \item [-e remount-ro] Change the behavior of the kernel code when errors are detected. Remount filesystem read-only.
        \item [/dev/sda\emph{N}] Device name.
        \item [-l] List the contents of the filesystem superblock (To check).
    \end{description}

    \section{Basic Configuration}
    

    According to informations placed in file /etc/inittab system uses 'tagets' instead of runlevels, so I'm changing default runlevel by taping ln -s /lib/systemd/system/<target name>.target 
/etc/systemd/system/default.target, where <target name> could be multi-user for runlevels 3 or graphical for runlevel 5. 

    \begin{lstlisting}[frame=single]
ln -sf /lib/systemd/system/multi-user.target /etc/systemd/system/default.target
    \end{lstlisting}
    To check all groups on system
    \begin{lstlisting}[frame=single]
cat /etc/group
    \end{lstlisting}
    To add new group 
    \begin{lstlisting}[frame=single]
groupadd students
    \end{lstlisting}
    To create new user, and add him to groups: [username], wheel (for possibility to use sudo), users and students. 
    \begin{lstlisting}[frame=single]
adduser -U -G wheel,users,students krzysiek
passwd krzysiek
id krzysiek # to check
    \end{lstlisting}
    To configure  DHCP Client
    \begin{lstlisting}[frame=single]
echo 'NETWORKING=yes' > /etc/sysconfig/network 
echo 'DEVICE=p2p1' > /etc/sysconfig/network-scripts/ifcfg-p2p1 
echo 'BOOTPROTO=dhcp' >> /etc/sysconfig/network-scripts/ifcfg-p2p1 
echo 'ONBOOT=yes' >> /etc/sysconfig/network-scripts/ifcfg-p2p1 
    \end{lstlisting}


\end{document}
